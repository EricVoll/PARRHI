

\tdplotsetmaincoords{60}{120} 
\begin{tikzpicture} [scale=0.05, tdplot_main_coords, axis/.style={->, black, thin}, 
vector/.style={-stealth,green,very thick}, 
robot/.style={green, very thick},
vector guide/.style={dashed,gray,thin}]

%standard tikz coordinate definition using x, y, z coords
\coordinate (O) at (0,0,0);

%tikz-3dplot coordinate definition using x, y, z coords


\pgfmathsetmacro{\axSize}{100}
\pgfmathsetmacro{\jointRadius}{30}

%Robot Points in cm (mm too large dimensions for library)
\coordinate (P0) at (0,0,0);
\coordinate (P1) at (5,0,33);
\coordinate (P2) at (5, 0, 77);
\coordinate (P3) at (15, 0, 100.5);
\coordinate (P4) at (47, 0, 100.5);
\coordinate (P5) at (55, 0, 100.5);
\coordinate (P6) at (63, 0, 100.5);
\coordinate (Point) at (5, 0 , 55);


%draw coordinate system axes
\draw[axis] (0,0,0) -- (\axSize,0,0) node[anchor=north east]{$x$};
\draw[axis] (0,0,0) -- (0,\axSize,0) node[anchor=north west]{$y$};
\draw[axis] (0,0,0) -- (0,0,\axSize) node[anchor=south]{$z$};

%draw the robot's joints and axes
\draw[robot] (P0) -- (P1); \fill[fill=gray] (P0) circle (\jointRadius pt);
\draw[robot] (P1) -- (P2); \fill[fill=gray] (P1) circle (\jointRadius pt);
\draw[robot] (P2) -- (P3); \fill[fill=gray] (P2) circle (\jointRadius pt);
\draw[robot] (P3) -- (P4); \fill[fill=gray] (P3) circle (\jointRadius pt);
\draw[robot] (P4) -- (P5); \fill[fill=gray] (P4) circle (\jointRadius pt);
\draw[robot] (P5) -- (P6); \fill[fill=gray] (P5) circle (\jointRadius pt);

%Draw PointRowot
\fill[fill=black] (Point) circle (\jointRadius pt);

%draw the 
\node[tdplot_main_coords,anchor=east]
at (P1){(Joint 1)};
\node[tdplot_main_coords,anchor=west]
at (P2){(Joint 2)};
\node[tdplot_main_coords,anchor=west]
at (Point){(Point scalar = 0.5)};
%\node[tdplot_main_coords,anchor=west]
%at (0,\ay,50){(0, \ay, 0)};
%\node[tdplot_main_coords,anchor=south]
%at (0,0,\az+50){(0, 0, \az)};
\end{tikzpicture}