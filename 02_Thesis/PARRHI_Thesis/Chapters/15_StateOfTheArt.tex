\chapter{State of the art}\label{Chap:StateOfTheArt}

\section{AR in Robotic applications}
\subsection{AR during development and testing}
Wolfgang Hönig et al. examined three different use cases of Augmented Reality \cite{hoenig2015mixed}. The research team primarily focused on the benefits offered by the co-existence of virtual and real objects during the development and testing phase. They documented three individual projects where the implementation of AR as a main feature resulted in lowered safety risks, simplified debugging and the possibility of easily modifying the actual, physical setup.

Another important dimension of development and testing is of financial nature: upfront investments and operating costs. For example, robotic aerial swarms tend to be quite expensive due to the high cost of drones. Hönig et al. successfully scaled up the number of objects in their swarms without adding physical hardware, thus, saving money and space \cite{hoenig2015mixed}. However, it is stated that this approach might not be applicable to all experiments since simulations are never perfect replicas of actual systems. This small delta in physical behaviour might be enough to raise doubts regarding the correctness of the experiment results.

All projects by Hönig et al. focus on isolating certain aspects of the system to analyse and test them
more flexibly, less expensively or with improved safety for all participants involved (humans and
machines). Similarly, Wünsche et al. have created a software framework for simulating certain parts
of robotic systems \cite{chen2009mixed}. These researchers from Auckland worked on methods to combine real world
and simulated sensor data and navigate a real-world robot in the combined environment.

\subsection{Operating robotic systems with AR support}
A well-known bottleneck in robotics is the controlling of and thus, the communication with robots. In all previously cited cases, Augmented Reality was not used to enhance the interaction between humans and robots but to mitigate the current challenges in developing robotic systems. Very early work by Milgram et al. \cite{milgram1993applications} shows that even the most basic implementations of AR technology, with the objective to improve the information exchange, enhance the bidirectional communication in multiple ways. The team proposed means to relieve human operators by releasing them from the direct control loop and using virtually placed objects as controlling input parameters. This replaces direct control with a more general command process.

As Gary Klein et al. stated, communicating intent is a key issue in effective collaboration within teams \cite{klein2005common}. Whenever robots and humans collaborate in a close manner, it is critically important to know each other’s plans or strategies in order to align and coordinate joint actions. For machines lacking anthropomorphic and zoomorphic features, such as aerial and industrial robots, it is unclear how to communicate the before-mentioned information in natural ways.

In order to solve this problem, Walker et al. \cite{walker2018communicating} explored numerous methods to utilize Augmented Reality to improve both efficiency and acceptance of robot-human collaborations via conveying the robot's intent. The group of researchers defined four methods of doing so, with varying importance being put on “information conveyed, information precision, generalizability and possibility for distraction” \cite{walker2018communicating}. The conclusion was that spatial Augmented Reality holograms are received much more intuitively than simple 2D projected interfaces.

\section{Parametrized Development in other disciplines}
%Example of Parametric Design: https://en.wikipedia.org/wiki/Parametric_design
Numerous disciplines utilize parametrized development environments heavily. For example the CAD software \textit{CADENCE} (integrated electrical circuits) offers parametrized cells to optimize the development process \cite{parametrizedCellElectricalInductor}. Users can place these cells and adjust certain parameters. Generally there is no coding skill required.

In architecture parametric design is taking overhand since the late 2000s. New capabilities in computer rendering and modelling opened up a whole new world for architects at the time. Using parametrized mathematical formulas with boundary constrains allows architects to generate building structures easier and more efficient \cite{stavric2011parametric}. Adapting to a changing environment during the design process is much easier, since a substantial portion of work can be completed by algorithms and software applications. Additionally the reusability of  components increases massively due to the very formal way of representation. 


 -> Pro
 -> Cons
\section{Current ways programming industrial robots}
 -> Pro
 -> Cons
\section{Parametrized AR in Robotic applications}