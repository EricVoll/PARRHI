\chapter{State of the Art}\label{Chap:StateOfTheArt}

This chapter highlights technologies and products which have contributed to this field of research. First, Augmented Reality will be explained and some important characteristics will be laid out. Then currently available AR-Standardisations will be described. Standards are  important since they help developers speed up their processes using predefined solutions and approaches. Further, the combined field of AR and Robotics will be described in detail: a)~how Augmented Reality technology is being applied, b)~which projects succeeded in doing so, and~c)~which benefits were realised. Then, a description of parametrised thinking/coding and other related disciplines that already utilise such an approach draws out a set of important learnings as the basis for the concept presented in Chapter 3. And finally, a brief overview is presented how this thesis tries to combine the above technologies and principles into an innovative way of programming and maintaining AR applications.

\section{Augmented Reality and its Standardisations}

This bachelor’s thesis revolves around AR and its use cases. It is important to clearly define some terms before using them throughout this document. AR technology must not be confused with Virtual Reality (VR). Whereas VR completely immerses the users, AR still allows the real world to be seen together with superimposed 3D objects projected into the user’s view. T. Azuma gave a very general definition of AR-Systems very early on~\cite{azuma1997survey}. He wrote that in order for a system to be classified as an AR system it has to fulfil three characteristics: 1) The system has to combine real and virtual objects, 2) it has to be interactive in real time, and 3) it has to spatially map physical and virtual objects to each other. The term Augmented Reality is often used as a synonym for the also commonly used expression “Mixed Reality”. 

There are different underlying principles in AR technology. Two types of AR technologies are in use today: 1) see-through displays with strong user immersion, and 2) monitor-based approaches like smartphones with cameras. A large variety of producers~\cite{ARProudcersVariety} release new AR-Devices in increasing frequency.

Standards often substantially contribute to the development and usage of technologies firstly, by unifying used technology stacks, thus potentially reducing complexity, and secondly by helping new members to navigate in a complex environment. There are not many international standards for AR due to the field’s emerging nature. Nevertheless, C. Perey et al.~\cite{perey2011current} examined existing standards and some best-practice methods for AR and identified some gaps in the AR value-chain pointing out interoperability problems between different components. Another conclusion was that there are some existing standards from other domains that could be used for AR, but that there is not enough agreement on which ones. The researchers lay out numerous standards for low-level implementations such as geographic location tracking, image tracking, network data transmission, etc.

Figueroa et al. \cite{figueroa2006conceptual} researched a "Conceptual Model and Specification Language for Mixed Reality Interface Components". They intended to lay a foundation for other developers to standardize 3D interface assets for others to build upon. The team developed an XML-based data structure called "3DIC" that defines the look and some minor behavioural actions of UI control elements. Although their work features useful components, it does not yet include certain features that robot-human AR interfaces need such as communicating with external machines. In addition, "3DIC" uses pseudo code that cannot be fully processed automatically, since pseudo-code is not executable. 

Concluding from this general overview on AR technology, there does not seem to be an existing standard for AR-HR-Interfaces utilizing parametrised Augmented Reality, that allows the development of complete HR-AR-applications that can be automatically processed, executed and used. 

\section{AR in Robotic Spplications}
\subsection{AR during Development and Testing of Robotic Systems}
Wolfgang Hönig et al. examined three different use cases of Augmented Reality~\cite{hoenig2015mixed}. The research team primarily focused on the benefits offered by the co-existence of virtual and real objects during the development and testing phase of robotic systems. They documented three individual projects where the implementation of AR as a main feature resulted in lowered safety risks, simplified debugging and the possibility of easily modifying the actual, physical setup.

Another important dimension of development and testing is of financial nature: upfront investments and operating costs. For example, robotic aerial swarms tend to be quite expensive due to the high cost of drones. Hönig et al. successfully scaled up the number of objects in their swarms without adding physical hardware by simulating additional drones in AR, thus, saving money and space~\cite{hoenig2015mixed}. However, it is stated that this approach might not be applicable to all experiments since simulations are never perfect replicas of actual systems. This small delta in physical behaviour might be enough to raise doubts regarding the correctness of the experiment results.

All projects by Hönig et al. focus on isolating certain aspects of the system to analyse and test them more flexibly, cheaply or with improved safety for all participants involved (humans and machines). Similarly, Chen et al. have created a software framework for simulating certain parts of robotic systems~\cite{chen2009mixed}. These researchers worked on methods to combine real world and simulated sensor data and navigate a real-world robot in the combined environment. Their approach was to intercept the raw sensor data originating from the real robot, and mixing it with the simulated data, before publishing it to the receivers.

For the to be built AR-Human Robot Interface, this approach could not only be used to combine the simulated virtual world and the real world data and then use it for the system's logic components but also to fully simulate the robot during the development of the system. This allows a decoupling of the development from the real world production hardware.

\subsection{Operating Robotic Systems with AR Support}
A well-known bottleneck in robotics is the controlling of and thus, the communication with robots~\cite{RoboticsScienceMag}. In all previously cited cases, AR was not used to improve the interaction between humans and robots but to mitigate the current challenges in developing robotic systems. Early work by Milgram et al.~\cite{milgram1993applications} shows that even the most basic implementations of AR technology with the objective to improve the information exchange, enhance the bidirectional communication between the humand and the machine in multiple ways. The team proposed means to relieve human operators by releasing them from the direct control loop and using virtually placed objects as controlling input parameters. This replaces direct control with a more general command process.

As Gary Klein et al. stated, communicating intent is a key issue in effective collaboration within teams~\cite{klein2005common}. Whenever robots and humans collaborate in a confined space, it is critically important to know each other’s plans or strategies in order to align and coordinate joint actions. For machines lacking anthropomorphic and zoomorphic features, such as aerial and industrial robots, it is unclear how the before-mentioned information can be communicated in natural ways between the humans and the machines.

In order to solve this problem, Walker et al. \cite{walker2018communicating} explored numerous methods to utilize AR to improve both the efficiency and the acceptance of robot-human collaborations via conveying the robot's intent. The group of researchers defined four methods of doing so, with varying importance being put on “information conveyed, information precision, generalizability and possibility for distraction”~\cite{walker2018communicating}. The conclusion was that spatial AR holograms are received much more intuitively than simple 2D projected interfaces.

For the framework to be built this implies that there have to be tools in place for the developer to explicitly communicate the robot's intent to the operator. There should also be ways to adapt the application's appearance to the domain in question (like industrial plants, laboratories or logistic centres), to maximise the acceptance of such interfaces.

\section{Parametrised Development}\label{Section:ParametricDesignIntoduction}
Some disciplines heavily utilize parametrised development environments, which allows to define the relationship between a process' intent to the outcome via parameters and rules. Adjusting the parameters automatically adjusts the outcome, which results in a high process-maintainability and agility. For example the CAD software package \textit{CADENCE} (used for designing integrated electrical circuits) offers parametrised cells to optimize the development process~\cite{parametrizedCellElectricalInductor}. Users can place these cells and adjust certain parameters. For example, spiral inductors can be configured via a simple UI that sets the parameters in the background. Aspects like outer dimensions, metal width, number of layers, etc. can be configured. The software then calculates its properties and behaviour at runtime. Generally, there is no coding skill required.

In architecture, Parametric Design has become increasingly prevalent since the late 2000s. New capabilities in computer rendering and modelling opened up a whole new world for architects at the time. Using parametrised mathematical formulas with boundary constraints allow architects to generate building structures more easily and more efficiently~\cite{stavric2011parametric}. Adaptions to a changing environment during the design process can be executed much more easily, since a substantial portion of the work can be completed by algorithms and software applications. Additionally, the reusability of components strongly increases due to the formal, abstract way of representation. Architecture studios are also beginning to include VR and AR technologies actively in their design processes to further incorporate and better understand this new design method called \textit{Parametric Design}~\cite{seichterDigitalDesignArch, salimSystemArchMR, wangFrameworkMXBIM}. 


\section{Current Methods Programming Industrial Robots}
To understand where and how the content of this research project fits into the world of robotics, this chapter describes the current workflows with robots and how they are configured and programmed. It is important to note that almost all industrial robots are programmed in three ways.
\begin{enumerate}
	\setcounter{enumi}{0}
	\setlength\itemsep{-1em}
	\item Teach Pendant
	\item Simulation / Offline Programming
	\item Teaching by Demonstration
\end{enumerate}

According to the British Automation \& Robot Association, over 90\% of all industrial robots are programmed using the Teach Pendant (TP) method~\cite{bara}. Basically, these devices are touch-tablets fitted for industrial use with emergency stop buttons, more durable materials, etc. They allow the operator to jog (a term for steering the robot manually using the TP) the robot into certain positions and offer a number of other possibilities. For simpler tasks, some manufacturers (e.g. Fanuc) offer specific User Interfaces where the operator simply enters parameters and positions. More complex goals can also be achieved with the Teach Pendant by programming in a textual manner, using each manufacturer's own language. This type of programming requires training and practice to achieve good results.

Teach Pendants are great for trivial and simple tasks, not requiring a lot of collaboration between factory components. Reprogramming the robot using this method leads to a partial downtime since the actual real robot has to be used. It is important to note that parametrised programming is already an industry standard in industrial robotics offered by several manufacturers. 

Offline programming is most often used for more complex tasks and production lines. While Offline Programming is very precise and powerful, it does require a substantial amount of training time for the operator. Teaching by Demonstration on the other hand, is the exact opposite. Most people succeed quickly, but the complexity of achievable tasks and the corresponding precision is limited.

The AR Robot-Human Interface proposed in this thesis could be categorised as Offline Programming, also interacts in the Teach Pendant mode, since the operator should be able to take over control of the robot when necessary. It can therefore be categorised as a hybrid approach.

\section{Combining Technologies to Mitigate AR Development Challenges}

The above technologies and principles can be combined in an attempt to mitigate challenges during AR development. 

Many industries are beginning to adopt AR in their daily processes. Its advantages and challenges are well known and being further developed. There are numerous examples of AR helping in different environments and tasks~\cite{DiegmannBenefitsAREdu, SalaminBenefitsAR, ARInMilRepair}. 

However, developing AR applications is generally-speaking still a very costly task. Applying the design principle of "Parametrisation" to AR might help lower the initial investment and the subsequent cost of changes and adaptions to altering environments. In addition, parametrisation offers the added benefit of enabling non-specialist developers to design AR use cases, thereby reducing the need for highly skilled and specialised programmers. Combining the benefits of both AR technology and parametrised architecture might therefore bear great potential.

In conclusion, there is prior work on most aspects of this bachelor’s thesis’ focus topic, which is a parametrised Augmented Reality Robot-Human Interface. Some of the discussed frameworks ('3DIC',~\cite{figueroa2006conceptual}) allow the textual definition of AR interfaces, but lack the ability to control robotic hardware and contain components like pseudo code that does not support an automated environment. Other projects showed the feasibility of combining simulated and real-world data to control robotic systems, but required programming by professional software engineers with experience in AR and robotics. In conclusion, there is no published research describing systems allowing simple parametrised representations of AR interfaces able to communicate with robots allowing more complex, higher-level applications to be built.












