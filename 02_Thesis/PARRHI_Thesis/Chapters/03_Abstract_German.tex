% !TeX spellcheck = de_DE
\chapter*{Kurzzusammenfassung}
\label{Chap:Abstract_German}
Die Entwicklung von Augmented Reality (AR) Anwendungen erfordert ein bestimmtes Set an Fähigkeiten, welches Spezialisten in unterschiedlichen Industrien oft nicht besitzen. Dies resultiert darin, dass AR Anwendungen entweder gar nicht oder langsamer entwickelt werden, oder, dass Software Entwickler beauftragt werden, was Kommunikationsaufwände, Kosten und möglicherweise die Entwicklungszeit negativ beeinflusst. Das präsentierte Konzept~"PARRHI" (Parametrisierte Augmented Reality Roboter Mensch Schnittstelle) versucht dieses Problem zu lösen, indem es Menschen, welche keine Erfahrungen in AR Entwicklung haben, ermöglicht industrielle AR Anwendungen zu entwickeln und auszuführen. Diese AR Anwendungen können mit industriellen Maschinen der jeweiligen spezifischen Industrie kommunizieren. Dies wird erreicht, indem ein abstraktes System über alle involvierten Technologien gespannt wird, welches Daten dem Entwickler durch Parameter zugänglich macht. Das System wurde in C\# und Unity am Beispiel eines Fanuc CR-i7A Roboters und einer Microsoft HoloLens implementiert. Eine Evaluierung untersuchte die Anwenderfreundlichkeit und Intuitivität der Entwicklungsumgebung anhand externer Versuchspersonen, welche eine Demo-Anwendung im PARRHI System implementierten. Die Evaluierung zeigte, dass das PARRHI System den Entwicklern sämtliche AR verwandten Herausforderungen erfolgreich abnimmt und die Entwicklungsprozesse vereinfacht. Allerdings kamen Probleme mit der Erweiterbarkeit des Systems zum Vorschein. Lösungsansätze zu den genannten Problemen werden am Ende der Arbeit vorgeschlagen.

Die parametrisierte Entwicklung von industriellen Augmented Reality Anwendungen, wie sie in diese Arbeit vorgeschlagen wird, scheint eine vielversprechende Herangehensweise zu sein, benötigt allerdings noch intensivere Forschungsarbeit und Entwicklung, bevor das Konzept in der Industrie zur Anwendung kommen kann.