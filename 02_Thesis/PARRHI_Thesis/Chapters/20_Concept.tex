\chapter{Concept}\label{Chap:Concept}

\section{Basic Overview}
To solve the stated problems (see section \ref{Section:ProblemDescription}) the to be implemented system from now on called Parametrised Augmented Reality Robot Human Interface (PARRHI) will have to fulfil these (very abstract) requirements:
\begin{itemize}
	\setlength\itemsep{-1em}
	\item Allow development of AR-HR Interfaces without knowledge in programming	
	\item Provide means to document the application's workflow
	\item Allow previous work to be reused
\end{itemize}



\section{V-Model}
To pinpoint the exact system requirements the V-Model is applied. 

Questions/Ideas:

Introduction -> Problem beschreiben 

State of the art
> andere parametrisierte Entwciklungsumgebungen
  > Pro and Cons aufzählen
  > Probleme aufzählen
> andere AR libraries
  > Pro and Cons aufzählen
  > Probleme und Lücken für meinen UseCase 
Schnittmengen diagramme
Oder per Tabellen die Anforderungen niederschreiben und "entwickeln"

Concept:
1.1 V-Modell
1.1.1 Requirements
1.1.2 Systemanforderungen
-> V-Modell hier beschreiben
Beschreibe warum ich die Architektur so und so gewählt habe
Komplett losgelöst von allen Tools und Sprachen etc.

Implementierung:
So habe ich es effektiv ausgeführt
Diese Sprachen dieses Tools. etc.

Evaluierung und Summary
Application 

\begin{enumerate}
	\setcounter{enumi}{0}
	\setlength\itemsep{-2em}
	\item I could write down some "User-Stories" -> no
	\item Or write down some Use-Cases -> no
	\item When do I start to get technical? (System I/O diagrams)
	\item Does the V-Model belong to the concept or is it implementation already?
	\item Is it ok to call the system PARRHI throughout the thesis?
\end{enumerate}


















