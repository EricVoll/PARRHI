\chapter{Introduction}\label{Chap:Introduction}

\section{Reasons for- and Challenges with Human-Robot Collaboration}\label{Section:ProblemDescription}

Today’s world is becoming increasingly automated in most aspects of our lives. One of the most important advancements in the last 30 years were robotic systems. They allow reducing costs while simultaneously increasing output, quality and efficiency. With the rising need for agile production plants, non-collaborative classic robotic systems do not seem to fulfil the modern requirements of Industry 4.0 any more, since they lack the ability to cooperate with humans at the required level, which is essential to combine the agility and intelligence of humans with the speed, precision and  power of robots. Current industrial robots’ lack the ability  of adapting quickly to changing environments. In contrast, the human’s strength lies in being able to learn and adapt quickly and flexibly. This is one of many reasons why Human-Robot (HR) collaboration is an inevitable step towards the future of robotics.

Robotics needed to adapt to a new situation where humans and robots share a common perimeter. Thus compliant robots were developed. It is not enough to be compliant and to not hurt humans for machines to act naturally, though. As Gary Klein et al. stated, communicating intent is a key issue in effective collaboration within teams~\cite{klein2005common}. 

Human-Machine communication with robotic systems – a critical requirement of modern robotics -  can be supported by Augmented Reality (AR). With rapid advancements in the field of robotics during the last few years~\cite{laschi2016soft}, innovative ways to program, develop and operate these highly complex systems need to be researched and developed. The field of AR has been a highly active field for the last 20 years, but the combination of AR and robotics has remained mostly untouched. Lately though, it has become a new focus area in research. While it is known that AR can improve communication in specific use cases~\cite{ARCommunicationBenefits}, actually developing them is complex due to multiple and highly diverse technical challenges needing to be resolved, e.g. 3D modelling, image tracking, performance and many more. Since most development environments require source code to be written for each use case, the reusability is quite low which may result in a lot of duplication of efforts by software Engineers producing highly specialised source code. Thus, the time-to-market is long and investments are significant, if not prohibitive for many smaller use cases.

According to the U.S. Department of Labour, there is a massive hunt for talent in the software industry~\cite{blsGov}. The department reported an expected growth of job employment in this industry of over 30\% within the next 10 years. One possible reason for the lack of talent could be that (similarly to many other engineering fields) acquiring the necessary skills to develop software applications takes a long time and is viewed as a challenging career choice. The lack of coding talent is also experienced in the robotics industry, which requires its developers to possess an even wider combination of skills, including software development, mechanical engineering and the basics of electrical engineering. The scarcity of developers has to the potential to slow down innovation and to extend the time-to-market. This in turn leads to a market premium of about 20\% for such developers~\cite{devShortageHackernoon}.

Furthermore, managers responsible for developing AR apps are professionals in their own fields, but might lack the detailed knowledge of how to develop AR applications. In addition, the highly interdisciplinary nature of such systems further contributes to the challenge of finding enough talent. 


\label{Section:PARRHIApproach}

In conclusion, AR applications can play a critical role in effective collaboration between humans and robots. Unfortunately, it is not easy to find enough qualified employees who have the domain-specific knowledge in developing AR applications. If non-programmers could set up, configure and develop Human-Robot interfaces involving AR and robot control all by themselves, i.e. without the need to write large amounts of source code and without being proficient in AR development, the cost efficiency and speed of development could be improved significantly.

However, the solution to the above challenge might lie in a strategy successfully used by many industries called \textit{parametrised development} (see \ref{Section:ParametricDesignIntoduction}), which has many great advantages. For example, changes can be made more easily, since only the input parameters have to updated, or such input parameters can also be  generated in specialised applications. Both can help mitigate the challenges involved in AR application development, making it easier for more and less specialised people to contribute.

This bachelor thesis aims to research the feasibility and practicability of such an abstract development environment which supports developing Augmented Reality interfaces for industrial applications without any specialised knowledge of AR. Chapter 2 describes the ”State of the Art” including current research and progress. Chapter 3 identifies gaps in current approaches and proposes a new concept. Chapter 4 describes an actual implementation in a physical robot/AR environment, and Chapter 5 evaluates the implementation’s performance.  Finally, Chapter 6 analyses certain aspects of generalising the concept for a wider spectrum of devices.

























