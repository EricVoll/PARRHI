\chapter{Introduction}\label{Chap:Introduction}

\section{Reasons for- and problems with Human-Robot collaboration}\label{Section:ProblemDescription}
%Bring examples of what could be fixed specifically (maintanance, dev etc.)
Today's world is highly automated in each and every aspect of our lives. One of the most important advancements within the last 30 years where robots. They allow us to minimize costs, while maximizing output, quality and efficiency. With the rising need of agile production plants, classic robotic systems do not seem to fulfil all requirements any-more. Industrial robots do have numerous abilities but adapting to new situations easily is certainly not one of them. In contrast, the human's strength always was exactly that - to learn quickly. Thus, Human-Robot (HR) collaboration is an inevitable step towards the future of humanity.

Robotics needed to adapt to the new situation where humans and robots share a perimeter, thus compliant robots where developed. For machines to feel natural it is not enough to be compliant and not hurt humans. As Gary Klein et al. stated, communicating intent is a key issue in effective collaboration within teams~\cite{klein2005common}. 

Augmented Reality (AR) can help us to communicate with robots in three dimensions. With rapid advancements in the field of robotics during the last few years~\cite{laschi2016soft}, innovative ways to program, develop and operate these highly complex systems need to be researched. Both robotics and AR have been heavily worked on for the last 20 years separately, but their combination has mostly remained untouched. Lately though, it has become a new focus area in research. While it is known that AR can improve communication in certain use cases~\cite{ARCommunicationBenefits}, actually developing them is not as easy, since there is a jungle of technologies and frameworks. Since most development environments require rely on source code to be written, the reusability is quite low, which may result in re-doing the same steps over and over again, requiring software engineers along the way to do everything. Thus, the time-to-market is long and expenses are high.

According to the U.S. Department of Labour, there is a massive hunt for talent in the software industry~\cite{blsGov}. The department reported an expected growth of over 30\% within the next 10 years. One possible reason could be, that acquiring the necessary skills to develop software applications takes a long time and is not viewed as an easy path. Specifically AR applications currently require a wide variety of skills. On top of that, many processes and steps have to be repeated for each new AR-application and it is increasingly hard to build upon other peoples work.

The lack of programming talent is also seen in the robotics industry, which requires its developers to posses an even wider set of skills, including software development, mechanical and the basics of electrical engineering. Due to the lack of developers the speed of innovation and time-to-market may either be slowed down or the companies will have to pay about 20\% above market value for their employees~\cite{devShortageHackernoon}.

\section{Parametrised development as an approach to solve AR-developer shortage}
\label{Section:PARRHIApproach}
Today, many industrial robots are only parametrised and not actually programmed, which results in lower requirements for employees and thus in less time and money spent on training. Employees only set certain locations, points and actions utilizing an easy to understand User-Interface. Applying a similar methodology to the development of Augmented Reality applications could lead to similar effects. If non-programmers could setup, configure and thus develop Human-Robot interfaces involving AR and robot controlling all by themselves, without the need to write a single line of code, the cost efficiency and speed of innovation can be improved.

This bachelor thesis aims to research the feasibility and practicability of such an abstract development environment. First there will be a "State of the Art" section, where current research and progress is displayed. From there some gaps will be identified, a concept proposed, implemented and tested in a real application.



























