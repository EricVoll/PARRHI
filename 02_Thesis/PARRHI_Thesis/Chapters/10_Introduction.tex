\chapter{Introduction}\label{Chap:Introduction}

Augmented Reality (AR) can help us to communicate with robots in three dimensions. With rapid
advancements in the field of robotics during the last few years \cite{laschi2016soft}, innovative ways to program, develop and operate these highly complex systems need to be researched. Both robotics and AR have been heavily worked on for the last 20 years, but their combination has mostly remained untouched. Lately though, it has become a new focus area in research.

According to the U.S. Department of Labour, there is a massive hunt for talent in the software industry~\cite{blsGov}. The department reported an expected growth of over 30\% within the next 10 years. One possible reason could be, that acquiring the necessary skills to develop software applications takes a long time and is not viewed as an easy path. Specifically AR applications currently require a wide variety of skills. On top of that, many processes and steps have to be repeated for each new AR-application and it is increasingly hard to build upon other peoples work.

The lack of programming talent is also seen in the robotics industry, which requires its developers to posses an even wider set of skills, including software development, mechanical and the basics of electrical engineering. Due to the lack of developers the speed of innovation and time-to-market is slowed as stated by multiple companies worldwide.

\section{Introduction to Augmented Reality}




