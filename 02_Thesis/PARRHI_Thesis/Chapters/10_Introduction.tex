\chapter{Introduction}\label{Chap:Introduction}

\section{Reasons for- and challenges with Human-Robot collaboration}\label{Section:ProblemDescription}
%Bring examples of what could be fixed specifically (maintanance, dev etc.)
Today's world is highly automated in each and every aspect of our lives. One of the most important advancements within the last 30 years were robots. They allow us to minimize costs, while maximizing output, quality and efficiency. With the rising need of agile production plants, non-collaborative classic robotic systems do not seem to fulfil all requirements any-more, since they lack the ability to cooperate with humans on the required level. Current industrial robots' main strength is not to adapt quickly to changing environments. In contrast, the human's strength always was exactly that - to learn quickly. This is one of many reasons why Human-Robot (HR) collaboration is an inevitable step towards the future of robotics.

Robotics needed to adapt to the new situation, where humans and robots share a perimeter, thus compliant robots were developed. It is not enough to be compliant and not hurt humans for machines to feel natural. As Gary Klein et al. stated, communicating intent is a key issue in effective collaboration within teams~\cite{klein2005common}. 

This crucial component (communication with robotic systems) can be supported by Augmented Reality (AR). With rapid advancements in the field of robotics during the last few years~\cite{laschi2016soft}, innovative ways to program, develop and operate these highly complex systems need to be researched. Also the field of AR has been heavily worked on for the last 20 years, but the combination of AR and robotics has mostly remained untouched. Lately though, it has become a new focus area in research. While it is known that AR can improve communication in certain use cases~\cite{ARCommunicationBenefits}, actually developing them is not as easy, since there is a big variety of challenges to solve. From 3D modelling to image tracking, performance and more. Since most development environments rely on source code to be written, the reusability is quite low, which may result in re-doing the same steps over and over again, requiring software engineers along the way to do everything. Thus, the time-to-market is long and expenses are high.

According to the U.S. Department of Labour, there is a massive hunt for talent in the software industry~\cite{blsGov}. The department reported an expected growth of job employment in this industry of over 30\% within the next 10 years. One possible reason for the lack of talent could be, that acquiring the necessary skills to develop software applications takes a long time and is not viewed as an easy path. The lack of programming talent is also seen in the robotics industry, which requires its developers to posses an even wider set of skills, including software development, mechanical and the basics of electrical engineering. Due to the scarcity of developers the speed of innovation and time-to-market may either be slowed down or the companies will have to pay about 20\% above market value for their employees~\cite{devShortageHackernoon}.

Furthermore, employees that might become responsible for developing such AR apps, might be professionals in their field (e.g. factory logistics), but do not know how to develop AR applications. This domain interdisciplinarity further contributes to the challenge of finding employees to reach one's AR specific goals. 

\section{Parametrised development to solve AR-development challenges}
\label{Section:PARRHIApproach}

In conclusion, AR applications can play a crucial role in effective collaboration between humans and robots. Unfortunately, it is not easy to find employees, who have  the domain specific knowledge required for the task and are capable of developing AR applications. If non-programmers could setup, configure and thus develop Human-Robot interfaces, involving AR and robot controlling all by themselves, without the need to write a single line of source code, the cost efficiency and speed of innovation can be improved.

Today, many industries use a strategy called \textit{parametrised development} (see \ref{Section:ParametricDesignIntoduction}), which has many great advantages. Amongst them are, that changes can be made relatively easy, since only the input parameters have to updated and one can also create easy to use application interfaces, which generate these parameters. Both mitigate the challenges involved in AR application development, making it easier for more people to contribute.

This bachelor thesis aims to research the feasibility and practicability of such an abstract development environment, which helps to develop Augmented Reality interfaces for robotic applications without any knowledge about AR itself. First there will be a "State of the Art" section, where current research and progress is displayed. From there some gaps will be identified, a concept proposed, implemented and tested in a real application.

%Applying a similar methodology to the development of Augmented Reality applications could lead to similar effects. 


























