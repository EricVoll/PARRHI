\chapter{Conclusion}\label{Chap:Conclusion}

This final chapter concludes this bachelor’s thesis. After summarising the project and its most important components, some future work is identified. 

\section{Summary}

Developing AR applications requires a certain set of skills including the knowledge of different frameworks, programming languages and engines. In industry, professionals of their field often do not have any or enough experience in AR development, which either results in no applications being developed, or Software Engineers being hired, which increases communication efforts, costs and possibly the time-to-market.

The presented framework called ”PARRHI” (Parametrised Augmented Reality Robot-Human Interface) tries to solve the problem of interdisciplinarity by enabling people who do not possess deep skills in AR development or programming in general to quickly and easily develop and execute AR applications for their specific domains.

The approach is to create an abstract layer above the technology layers, able to handle the AR and real-world communication components. The developer, who is a person with the specific domain knowledge wanting to create AR applications, can now focus on the application’s workflow and let the PARRHI system manage all other aspects like image tracking, AR, wireless communication with the real world and more.

To further support the developer, the PARRHI system follows a concept called ”parametric thinking”. This means that the program written by the developer is parametrised using placeholders. These placeholders can then be assigned data which the PARRHI system discloses to the program, or also other objects defined in the parametrised program. 

In conclusion, the developer’s required skill set is limited to their specific domain knowledge. The PARRHI system handles all other aspects of AR Human Robot Interfaces on its own.

\section{Future Work}\label{Section:FutureWork}

Although the proposed PARRHI system seems to be an interesting and promising approach, it is by all means far from being a highly advanced implementation. As its evaluation showed, there still are some problems in terms of user-friendliness that need to be addressed. An industrial implementation of PARRHI therefore needs further development before successful deployment.

Before undertaking further development, a thorough evaluation with a higher quality standard and many more participants that fit into the role of the developer has to be performed. In addition, the requirements should be revised and documented formally in cooperation with industry partners to verify their applicability. Using the new knowledge as a starting point, certain abstractions would have had to be made to the core concept of the PARRHI system.












