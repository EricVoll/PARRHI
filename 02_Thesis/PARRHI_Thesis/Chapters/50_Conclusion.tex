\chapter{Conclusion}\label{Chap:Conclusion}

This final chapter will conclude this bachelor's thesis. After summarising the project and its biggest components, I will propose some solutions to solve the evaluated problems from section~\ref{Chap:Evaluation}. After that, a section about a possible technology transfer will follow, which describes how the PARRHI system could be fitted to other hardware (robots, factory machines, AR devices), programming languages or also 3D game engines. The very last section outlines some future projects, which could build upon this bachelor's thesis.

\section{Summary}

Developing Augmented Reality applications needs a certain set of skills, which involves knowledge in different frameworks, programming languages and engines. In the industry professionals of their field often do not have any experience with AR development, which either result in no applications being developed, or Software Engineers being hired, which increases communication efforts, costs and possibly the time to market. 

The presented system called "PARRHI" (Parametrised Augmented Reality Robot-Human Interface) tries to solve the problem of interdisciplinarity by enabling people, who do not have any skills towards AR development or programming in general, to quickly and easily develop and execute AR applications for their specific domain. 

The approach is, to create an abstract layer above all technicalities, that handle the Augmented Reality and real-world communication components. The developer, which is the person with the specific domain knowledge, who wants to create such an AR application, now can focus on the application's workflow, and let the PARRHI system manage all other aspects like image tracking, AR, wireless communication with the real world and more.

To further support the developer, the PARRHI system follows a concept called "parametric thinking". This means, that the program, which is written by the developer, is parametrised using placeholders. These placeholders can now be data which the PARRHI system discloses to the program, or also other objects defined in the parametrised program. 

In conclusion, the developer's necessary skill set is limited to their specific domain knowledge. The PARRHI system handles all other aspects of AR Human Robot Interfaces on its own.

\section{Future Work}\label{Section:FutureWork}

Although the PARRHI system seems to be an interesting and promising approach, it is by all means not a perfect implementation. As the evaluation showed, there are some problems that exist in an academic environment. These problems are only expected to scale when the system is brought into the industry. 

Before undertaking further development, a thorough evaluation with a higher quality standard and many more participants that fit into the role of the developer has to be done. In addition to that, the requirements have to be revised and documented formally in cooperation with industry partners to verify their correctness. Using the new knowledge as a starting point, certain abstractions would have had to be made to the core concept of the PARRHI system.

From there on, further research has to be done and new prototypes developed.











