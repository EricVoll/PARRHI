\chapter*{Abstract}\label{Chap:Abstract}

Developing industrial Augmented Reality (AR) applications requires a certain set of skills which professionals of their field often do not have. This either results in fewer or less quick AR applications being developed, or Software Engineers being hired, which increases communication efforts, costs and possibly the time-to-market. The presented framework "PARRHI" (Parametrised Augmented Reality Robot-Human Interface) tries to solve this problem by enabling people who do not possess deep AR developing experience to develop and execute AR applications that are capable of communicating with industrial machines for their specific domains. This is done by creating an abstract layer above involved technologies and disclosing data to the developer through parameters. The framework was implemented in C\# and Unity using a Fanuc CR-i7A Robot as a machine-demonstrator and a Microsoft HoloLens as an AR device. The framework's implementation has been evaluated against usability and intuitiveness by external participants posing as developers. The evaluation showed that the PARRI framework successfully relieves the developer from all AR related challenges and simplifies the development workflow, but has its limitations with extendibility. Solutions to these problems are proposed. 

The parametrised development of industrial Augmented Reality applications as proposed in this thesis seems to be a promising approach but needs further research and work before it can be deployed to the industry.