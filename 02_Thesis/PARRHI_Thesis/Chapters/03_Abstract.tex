\chapter*{Abstract}\label{Chap:Abstract}

Developing industrial Augmented Reality (AR) applications requires a certain set of skills which professionals of their field often do not have. This either results in no AR applications being developed, or Software Engineers being hired, which increases communication efforts, costs and possibly the time-to-market. The presented framework "PARRHI" (Parametrised Augmented Reality Robot-Human Interface) tries to solve this problem by enabling people who do not possess deep AR developing experience to easily develop and execute AR applications that are capable of communicating with industrial machines for their specific domains. This is done by creating an abstract layer above involved technologies and disclosing data to the developer through parameters. The framework was implemented in C\# and Unity using a Fanuc CR-i7A Robot as a demonstrator and a Microsoft HoloLens as an AR device. The implemented system was evaluated using two different methods. One compared traditional AR development workflows with the PARRHI system, and the other evaluated the systems usability and intuitiveness with external participants posing as developers. The PARRI system successfully relieves the developer from all AR related challenges, but has its limitations with extendibility. Both evaluations showed conceptual and usability problems which are addressed by proposed solutions.

The parametrised development of industrial Augmented Reality applications as proposed in this thesis seems to be an interesting approach but needs further research and work to be deployed.
