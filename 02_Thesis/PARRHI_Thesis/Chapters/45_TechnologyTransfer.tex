\chapter{Technology Transfer}
\label{Chap:TechnologyTransfer}
\label{Section:TechnologyTransfer}

This chapter will describe what measurements had to be undertaken, in order to extend the PARRHI system as it is, for a wider variety of use cases and functionalities. The structure of this section will be grouped by technologies or components within the PARRHI implementation.

\subsection{Replacing the Target Machine}
In the provided implementation a Fanuc CR-7iA/L robot was chosen as a demonstrator for the PARRHI concept. Of course, there is a big variety of desirable targetable machines, like other robot types from Fanuc or other suppliers, other machines like industry 4.0 machines, etc.

The PARRHI library is written in a very modular manner. This means, that extending or replacing certain parts of it is very easy. To replace the target machine, one would only have to extend the Input/Output modules and the Real World Model, to be able to communicate with the desired machine.

At this point, I have to separate three different cases. Case 1 would be to communicate with another robot in the Fanuc product family. Case 2 is to communicate with an industrial robot of another brand and finally case 3 would be to include machines which are not industrial robots. 

Case 1 is very easily achievable since we built a working communication system for Fanuc controllers. In this case, only the Real World Model has to be updated, since the forward kinematics is based on this specific robot. The modification to involve other Fanuc robots would thereby only mean, to adapt the forward kinematic model's vectors and matrices as defined in section~\ref{Section:ForwardKinematics}. Currently, this model is defined in source code, since in my specific implementation example I only had one robot to work with. So a generalisation would have been a slight overkill. Theoretically, one could simply expose the forward kinematics model via an Interface, and let knowledgable developers implement their own robot. 

Case 2 is a bit more tricky since the communication problem has to be solved again. One could reuse the Robot Library (see section~\ref{Section:RobotLibrary}), and only replace the other endpoint of the network socket with the new robot. The controlling and parsing of the commands on the robot side would have to be rewritten on the specific system. Our current KAREL / TP solution only works on Fanuc Robot-Controllers and not on any other type of industrial robot. Then again the forward kinematic's model would need an adaption to the new robot too.

Case 1 and Case 2 do not require any changes to the PARRHI system itself. The Parametrised Program could be used as it is. In case 3 though, this would likely change, since conceptually another machine is not the same as an industrial robot. The Distance Trigger, for example, would not make any sense if the target machine does not have any moving parts. This means, that in addition to the communication module (Input / Output Modules), one would probably have to define new triggers and actions for the target machine, what requires changes to the PARRHI library itself. Since the PARRHI library is implemented with element base classes and inheritance structures, the implementation of custom triggers and actions is rather easy.

\subsection{Replacing the AR / 3D Engine}

The current implementation uses the Unity Game Engine as a host. Since all the application logic (PARRHI Library) and also the Robot Library is implemented in engine independent .NET libraries, one could easily switch to any other 3D engine, which is capable of building for Augmented Reality Devices and is able to include .NET dynamically linked libraries (DLLs). 

The new implementation would have to implement three main aspects: Firstly, the engine has to host the PARRHI and Robot libraries, which is easy, as long as the engine's scripting environment supports .NET libraries. Secondly, the AR-Output Module has to be rewritten for the new engine. In my implementation, this only took about 30 lines of code. The AR-Output module only instantiates new Holograms in the engine's environment and displays them accordingly. Lastly, the image tracking problem would have had to be solved again. The used image tracking library (Voforia engine, see section~\ref{Section:ImageTracking}) supports the Unity Engine, iOS, Android and UWP platforms. In principle, any library could be used.

For example, the Unreal Engine 4 \cite{unrealEngine} allows $C\#$ DLLs to be imported, and can thus work easily with the PARRHI libraries.

\subsection{Replacing AR Device}

Transitioning to another AR device is probably the easiest of all technology transitions. Unity itself supports building for multiple devices including mobile phones, desktop apps, HoloLens, consoles and more~\cite{UnityPlatforms}. Basically, the new device only has to possess an accessible camera, which can be used for AR applications, and there has to be a 3D engine, which supports the device. 